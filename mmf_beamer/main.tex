% Welcome! This is the template for beamer presentations for the department of mathematics and mechanics

% See README.md for more informations about this template.

% This style has been developed as part of the "Style manual" project.


% We defined three theme colours: MMFGreen, MMFgray, MMFblack
% For example, to write some text in official colour
% just use: \textcolor{MMFGreen}{Text}

% Note that [usenames,dvipsnames] is MANDATORY due to compatibility
% issues between tikz and xcolor packages.

\documentclass[usenames,dvipsnames,]{beamer}
\usepackage[utf8]{inputenc}
\usepackage{verbatim}
\usepackage[english,russian]{babel}
\usetheme{MMF}

%Translations into russian, to use, uncomment and add "russian" to babel
%\newtranslation[to=russian]{Theorem}{Теорема}
%\newtranslation[to=russian]{Example}{Пример}
%\newtranslation[to=russian]{Corollary}{Следствие}
%\newtranslation[to=russian]{Lemma}{Лемма}
\newtranslation[to=russian]{Abstract}{Abstract}
\renewcommand\proofname{Proof}
%\renewcommand{abstract}{Abstract}
%%% Bibliography
\usepackage[style=authoryear,backend=biber]{biblatex}
\addbibresource{bibliography.bib}

% Author names in publication list are consistent 
% i.e. name1 surname1, name2 surname2
% See https://tex.stackexchange.com/questions/106914/biblatex-does-not-reverse-the-first-and-last-names-of-the-second-author
\DeclareNameAlias{author}{first-last}

%%% Suppress biblatex annoying warning
\usepackage{silence}
\WarningFilter{biblatex}{Patching footnotes failed}

%%% Some useful commands
% pdf-friendly newline in links
\newcommand{\pdfnewline}{\texorpdfstring{\newline}{ }} 
% Fill the vertical space in a slide (to put text at the bottom)
\newcommand{\framefill}{\vskip0pt plus 1filll}
%\newcommand\montserratfont[1]{{\usefont{T1}{montserrat}{bx}{n} #1 }}

\title[DMM Beamer Theme]{On The Explosion of Large Death Stars}
\date[May 1977]{May 25, 1977}
\author[Luke Skywalker]{
  Luke Skywalker, Ph.D.
  \pdfnewline
  \color{MMFgray}{\texttt{skywalker3@g.nsu.ru}}
}
\institute{Department of Mathematics and Mechanics, 

Novosibirsk State University}

\begin{document}

\begin{frame}
\titlepage
\end{frame}

\begin{frame}{Outline}
\tableofcontents
\end{frame}

\section{Before you start}
\begin{frame}{Overleaf users}

\begin{alertblock}{Warning}
You can ignore this slide if you're \textbf{not} working with Overleaf.
\end{alertblock}

\vskip 0.5cm

Overleaf, Beamer and Biber do not always get along well together. For this reason, if you make a mistake while writing this presentation, in the drop-down error message you'll \textbf{always} get Biber-related error messages.

\vskip 0.5cm

Luckily, you just have to click on ``\texttt{go to first error/warning}'' and the UI will scroll to the line containing your mistake.

\end{frame}

\begin{frame}[fragile]
\frametitle{Compiling}

\begin{alertblock}{Warning}
You can ignore this slide if you're working with Overleaf.
\end{alertblock}

To compile this deck you'll need the \texttt{biber} package. Probably your \TeX editor already supports it; if not, you will easily find online the instructions to install it.

\vskip 0.5cm

If you're not using an editor, you can compile this presentation using the command line by running:

\begin{verbatim}
$ pdflatex main.tex
$ biber main.bcf
$ pdflatex main.tex
$ pdflatex main.tex
\end{verbatim}


\end{frame}

\section{Colors}

\begin{frame}{Colors}

For this template we defined four colors, following the Style Manual of the University of Udine:
\begin{itemize}
\item \textcolor{white}{\marker{\texttt{MMFGreen}}}
\item \textcolor{white}{\marker[MMFgray]{\texttt{MMFgray}}}
\item \textcolor{white}{\marker[MMFblack]{\texttt{MMFblack}}}
\end{itemize}

\vskip 0.5cm

You can use these colors as you want in your presentation. For example, you can \textbf{\textcolor{MMFGreen}{color the text in green}} by writing \texttt{\textbackslash\{MMFGreen\}\{Text\}}.

\vskip 0.5cm

We also redefined many of the most common \LaTeX{} and Beamer commands, like \texttt{itemize}, \texttt{block}, etc. You will see samples of these commands in the following slides.

\end{frame}

\section{Blocks}

\begin{frame} 
\frametitle{This is a page with a title and a subtitle} 
\framesubtitle{And also some blocks.} 
\begin{block}{Goal of the mission}
Shoot in the Death Star's exhaust port and destroy it before the it can fire on the Rebel base.
\end{block} 
\begin{alertblock}{Take care!}
TIE Fighters may chase you while approaching the target.
\end{alertblock} 
\begin{exampleblock}{Use the force you must}
Remember your training with Obi-Wan, and use the Force to make the perfect shoot.
\end{exampleblock} 

\end{frame}

\section{Enumerates, itemizes and description}

\subsection{Enumerates and itemizes}

\begin{frame}{Enumerates and itemizes}

This is an example of \texttt{itemize}.
\begin{itemize}
	\item A long time ago in a galaxy far, far away...
	\item There once was a man
	\begin{itemize}
	    \item His name was John
	\end{itemize}
\end{itemize}
And this is an example of \texttt{enumerate}.

\begin{enumerate} 
  \item Go to the Death Star.
  \item Find the exhaust port.
  \item Make the perfect shot.
  \begin{enumerate}
      \item shoot
      \item make it perfect
  \end{enumerate}
  \item Become a hero.
\end{enumerate}
\end{frame}

\subsection{Description}

\begin{frame}[fragile]
\frametitle{Description}
This is an example of \texttt{description}.

\begin{description}
\item<2->[Vader] \emph{I am} your father.
\item<1->[Luke] No. No! That's not true! \textbf{That's impossible!}
\end{description}

\begin{uncoverenv}<3>
  \vskip 0.5cm
  And while we're here, let's have a look to \texttt{verbatim} as well, to see how we made items appear in arbitrary order:
  \vskip 0.5cm
  \begin{verbatim}
\begin{description}
  \item<2->[This is the first item] one
  \item<1->[This is the second item] two
\end{description}
  \end{verbatim}
\end{uncoverenv}

\end{frame}

\section{Maths}

\begin{frame}{Formulas}
The formulas look as follows
\begin{center}
 $x^2 + y^2 = z^2$
 $$\iiint\limits_V div\  \mathbf{F}\  dV=\iint\limits_S \langle\mathbf{F},\mathbf{n}\rangle dS$$
 $$L_{ij}=\frac{1}{2}\left[ \sum_k \frac{\partial x^k}{\partial \xi^i}\frac{\partial x^k}{\partial \xi^j}- \delta_{ij} \right]$$
\end{center}

You can also number the equations
\begin{equation}
|G|=|H||G:H|
\end{equation}

\begin{equation}
e^{i\pi}+1=0 \tag{a custom tag}
\end{equation}

\vskip 0.5cm

%If you want to use the default \LaTeX{} math fonts, just go to \texttt{beamerfontthemeuniud.sty} and uncomment the line containing `\texttt{\textbackslash usefonttheme[onlymath]\{serif\}}'.

\end{frame}

\begin{frame}{Theorems}

Blocks for \texttt{theorem}, \texttt{corollary}, \texttt{definition}, \texttt{definitions}, \texttt{fact}, \texttt{example} и \texttt{examples} are also available.

\begin{theorem}
There exists and infinite set
\end{theorem}
\begin{proof}[Proof]
This follows from the axiom of infinity
\end{proof}
\begin{example}[Natural]
The set $\mathbb{N}$ of natural numbers is infinite.
\end{example}

\end{frame}

\section{Other blocks}

\begin{frame}{Other blocks}

Here we display examples of \texttt{abstract}, \texttt{verse}, \texttt{quotation}, and \texttt{quote}.

\vskip 0.5cm

\begin{abstract}
This is an abstract.
\end{abstract}
\begin{verse}
This is a verse.
\end{verse}
\begin{quotation}
This is a quotation.

\raggedleft -Han Solo
\end{quotation}
\begin{quote}
A quote this is.

\raggedleft -Yoda
\end{quote}

\end{frame}

\section{Bibliography and Publications}
\begin{frame}[fragile]
\frametitle{Bibliography}

You can cite an article
\begin{itemize}
\item normally using \texttt{\textbackslash cite}, e.g.: (\cite{article1})
\item or display the full citation using \texttt{\textbackslash fullcite}, e.g.:  \fullcite{article1}
\end{itemize}

\vskip 0.5cm
Look at the code of the following slide to see how to automatically split the bibliography on many slides. You can also use \texttt{\textbackslash nocite\{*\}} to display the non-cited publications as well.

\end{frame}

\begin{frame}[t,allowframebreaks]
\frametitle{Bibliography}

\nocite{*} % will display the non-cited publications as well. Useful for a publication list.

\printbibliography

\end{frame}

\section{Additional commands}

\begin{frame}[fragile]
\frametitle{Framecard}
You can use \texttt{\textbackslash framecard}, to make a frame with text on a colored background
 \texttt{\textbackslash framecard}.
\vskip 0.5cm 
For instance you may write
\begin{verbatim}
\framecard{A SECTION TITLE}
\end{verbatim}

To display the phrase on a green background 
You can also change the font by writing
\begin{verbatim}
\framecard{A SECTION
TITLE}
\framecard[Color_name]{A SECTION TITLE\\
WITH THE NICE COLOR}
\end{verbatim}
Here is how it will look like

\end{frame}

\framecard{A SECTION TITLE}
\framecard[UniBlue]{A SECTION TITLE\\
WITH THE NICE COLOR}

\begin{frame}[fragile]
\frametitle{Framepic}

%You can display a frame with a background image using the command \texttt{\textbackslash framepic}. The image will be \textbf{adapted vertically} to fit the the frame. 


For example, you can write:
\begin{verbatim}
\framepic{graphics/darth}{
	\framefill
    \textcolor{white}{Luke,\\I am your supervisor}
    \vskip 0.5cm
}
\end{verbatim}

Alternatively, to make the background 50\% transparent, you can write \texttt{\textbackslash framepic[0.5]\{graphics/darth\}...}


You can see the results of the commands above in the following slides.

\end{frame}


\framepic{graphics/darth}{
	\framefill
    \textcolor{white}{Люк,\\Я твой научрук}
    \vskip 0.5cm
}

\framepic[0.5]{graphics/darth}{
	\framefill
    \textcolor{white}{Люк,\\Я твой научрук}
    \vskip 0.5cm
}


\begin{frame}[t,fragile,allowframebreaks]
\frametitle{Other bonus commands}

We provide two other bonus commands:
\begin{description}
\item[\texttt{pdfnewline}] you can use \texttt{\textbackslash pdfnewline} to avoid the annoying \texttt{hyperref} related warnings when using newlines in the document's title, author, etc. For example, in this presentation the author is defined as:
\begin{verbatim}
\author[Luke Skywalker]{
  Luke Skywalker, Ph.D.
  \pdfnewline
  \texttt{skywalker3@g.nsu.ru}
}
\end{verbatim}
\item[\texttt{marker}] you can use \texttt{\textbackslash marker} to highlight some text. The default color is \marker{orange}, but you can also \marker[UniBlue]{use a custom color}. For example:
\begin{verbatim}
\marker{Default color}
\marker[UniBlue]{Custom Color}
\end{verbatim}
\item[\texttt{framefill}] you can use \texttt{\textbackslash framefill} to put the text at the bottom of a slide by filling all the vertical space.
\end{description}

\end{frame}

\begin{frame}
\frametitle{С русским текстом!}\framesubtitle{Лапа папала}
Лалалалала
\end{frame}
\end{document}